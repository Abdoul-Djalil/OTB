% We want this material to fit on two pages
\small

\chapter*{About the Cover}

Creating the cover image demonstrating the capabilities of the toolkit was a
challenging task.\footnote{The source code for the cover is available from
InsightDocuments/SoftwareGuide/Cover/Source/.} Given that the origins of ITK
are with the Visible Human Project it seemed appropriate to create an image
utilizing the VHP data sets, and it was decided to use the more recently
acquired Visible Woman dataset.  Both the RGB cryosections and the CT scans
were combined in the same scene.

\begin{description}

\item [Removing the Gel.]
The body of the Visible Woman was immersed in a block of gel during the
freezing process. This gel appears as a blue material in the cryogenic data.
To remove the gel, the joint histogram of RGB values was computed. This
resulted in an 3D image of $256\times256\times256$ pixels. The histogram
image was visualized in VolView.\footnote{VolView is a commercial product
from Kitware. It supports ITK plug-ins and is available as a free viewer or
may be licensed with advanced functionality. See
http://www.kitware.com/products/volview.html for information.} The cluster
corresponding to the statistical distribution of blue values was identified
visually, and a separating plane was manually defined in RGB space. The
equation of this plane was subsequently used to discriminate pixels in the
gel from pixels in the anatomical structures. The gel pixels were zeroed out
and the RGB values on the body were preserved.

\item[The Skin.]
The skin was easy to segment once the gel was removed. A simple region
growing algorithm was used requiring seed points in the region previously
occupied by the gel and then set to zero values. An anti-aliasing filter was
applied in order to generate an image of pixel type float where the surface
was represented by the zero set. This data set was exported to VTK where a
contouring filter was used to extract the surface and introduce it in the VTK
visualization pipeline.

\item[The Brain.]
The visible part of the brain represents the surface of the gray matter.  The
brain was segmented using the vector version of the confidence connected
image filter.  This filter implements a region growing algorithm that starts
from a set of seed points and adds neighboring pixels subject to a condition
of homogeneity.

The set of sparse points obtained from the region growing algorithm was
passed through a mathematical morphology dilation in order to close holes and
then through a binary median filter. The binary median filter has the
outstanding characteristic of being very simple in implementation by applying
a sophisticated effect on the image. Qualitatively it is equivalent to a
curvature flow evolution of the iso-contours. In fact the binary median
filter as implemented in ITK is equivalent to the majority filter that
belongs to the family of voting filters classified as a subset of the
\emph{Larger than Life} cellular automata. Finally, the volume resulting from
the median filter was passed through the anti-aliasing image filter. As
before, VTK was used to extract the surface.

\item[The Neck Musculature.]
The neck musculature was not perfectly segmented. Indeed, the resulting
surface is a fusion of muscles, blood vessels and other anatomical
structures. The segmentation was performed by applying the
VectorConfidenceConnectedImageFilter to the cryogenic dataset. Approximately
60 seed points were manually selected and then passed to the filter as
input. The binary mask produced by the filter was dilated with a mathematical
morphology filter and smoothed with the BinaryMedianImageFilter. The
AntiAliasBinaryImageFilter was used at the end to reduce the pixelization
effects prior to the extraction of the iso-surface with vtkContourFilter.

\item[The Skull.]
The skull was segmented from the CT data set and registered to the cryogenic
data. The segmentation was performed by simple thresholding, which was good
enough for the cover image. As a result, most of the bone structures are
actually fused together. This includes the jaw bone and the cervical
vertebrae.

\item[The Eye.] 
The eye is charged with symbolism in this image. This is due in part because
the motivation for the toolkit is the analysis of the Visible Human data,
and in part because the name of the toolkit is \emph{Insight}.

The first step in processing the eye was to extract a sub-image of
$60\times60\times60$ pixels centered around the eyeball from the RGB
cryogenic data set. This small volume was then processed with the vector
gradient anisotropic diffusion filter in order to increase the homogeneity of
the pixels in the eyeball.

The smoothed volume was segmented using the
VectorConfidenceConnectedImageFilter using 10 seed points. The resulting
binary mask was dilated with a mathematical morphology filter with a
structuring element of radius one, then smoothed with a binary mean image
filter (equivalent to majority voting cellular automata). Finally the mask
was processed with the AntiAliasBinaryImageFilter in order to generate a
float image with the eyeball contour embedded as a zero set.

\item[Visualization.]
The visualization of the segmentation was done by passing all the binary
masks through the AntiAliasBinaryImageFilter, generating iso-contours with
VTK filters, and then setting up a VTK Tcl script. The skin surface was
clipped using the vtkClipPolyDataFilter using the implicit function
vtkCylinder. The vtkWindowToImageFilter proved to be quite useful for
generating the final high resolution rendering of the scene ($3000\times3000$
pixels).

\item[Cosmetic Postprocessing.]
We have to confess that we used Adobe Photoshop to post-process the image. In
particular, the background of the image was adjusted using Photoshop's color
selection. The overall composition of the image with the cover text and
graphics was also performed using Photoshop.

\end{description}

\normalsize
