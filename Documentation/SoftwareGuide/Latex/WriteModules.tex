\chapter{Adding New Modules}
\label{chapter:newModules}

This chapter is concerned with the creation of new modules. 
The following sections give precise instructions about :
\begin{itemize}
	\item the organization of your directories
	\item the files you need to include
	\item what they must contain
	\item ...
\end{itemize}

\section{How to Write a Module}
\label{sec:writemodule}

There is a template of OTB remote module which help you  start developing you're
remote module: \href{https://github.com/orfeotoolbox/otbExternalModuleTemplate}{External Module Template}

Each module is made of different components, which are described in the following sections.

\section{The otb-module.cmake file}

This file is mandatory. It follows the CMake syntax, and has two purposes: 

\begin{itemize}
       \item Declare dependencies to other modules, 
       \item Provide a short description of the module purpose. 
\end{itemize}

These purposes are fulfilled by a single CMake Macro call: 

\begin{verbatim}
otb_module(TheModuleName DEPENDS OTBModule1 OTBModule2 ... OTBModuleN DESCRIPTION "A description string")
\end{verbatim}

\textbf{Note}: You can use the keyword TEST\textunderscore DEPENDS to declare module dependencies that only applies to the tests.

\section{The CMakeLists.txt file}

The CMakeLists.txt file is mandatory. It contains only a few things.
First, it declares a new CMake project with the name of the module. 

\begin{verbatim}
project(TheModuleName)
\end{verbatim}

Second, if the module contain a library (see src folder section below), it initializes the TheModuleName\textunderscore LIBRARIES CMake variable (if your module only contains headers or template code, skip this line): 

\begin{verbatim}
set(TheModuleName_LIBRARIES OTBTheModuleName)
\end{verbatim}

You can build your remote modules inside the OTB source tree by copying your
source inside the directory \textit{Module/Remote} or against an existing OTB
build tree (note that it does not work with an install version of OTB). 

The configuration below will handle both cases and take care of all the CMake
plumbing of the module:

\begin{verbatim}
  if(NOT OTB_SOURCE_DIR)
    find_package(OTB REQUIRED)
    list(APPEND CMAKE_MODULE_PATH ${OTB_CMAKE_DIR})
    include(OTBModuleExternal)
  else()
    otb_module_impl()
  endif()
\end{verbatim}

The overall file should look like this:
 
\begin{verbatim}
cmake_minimum_required(VERSION 2.8.9)
project(TheModuleName)
set(ExternalTemplate_LIBRARIES OTBTheModuleName)

if(NOT OTB_SOURCE_DIR)
    find_package(OTB REQUIRED)
    list(APPEND CMAKE_MODULE_PATH ${OTB_CMAKE_DIR})
    include(OTBModuleExternal)
  else()
    otb_module_impl()
  endif()
\end{verbatim}

\section{The include folder}

The include folder will contain all your headers (*.h files) and template method boy files (*.txx or *.hxx). It does not require any additional file (in particular, no CMakeLists.txt file is required). 

\section{The src folder }

The src folder contains the internal implementation of your module : 

\begin{itemize}
       \item  It typically contains cxx source files that will be compiled into a library.  
       \item  It can contain header files for classes used only within the implementation files of your module. Any header file present in the src folder will not be installed, and will not be available to other modules depending on your module.
\end{itemize}

If your modules is made of template only code, you do not need a src folder at all.

If present, the src folder requires a CMakeLists.txt file.

The first part of the CMakeLists.txt file is classical, as it builds the library and links it: 

\begin{verbatim}
set(OTBTheModuleName_SRC
    sourceFile1.cxx
    sourceFile2.cxx
    sourceFile3.cxx
    ...
    sourceFileN.cxx)

add_library(OTBTheModuleName ${OTBTheModuleName_SRC})

target_link_libraries(OTBTheModuleName ${OTBModule1_LIBRARIES} ${OTBModule2_LIBRARIES} ... ${OTBModuleN_LIBRARIES})
\end{verbatim}

\textbf{Notes}: 

\begin{itemize}
       \item  Library name should match the one declared in the root CMakeLists.txt when setting CMake variable TheModuleName\textunderscore LIBRARIES,   
       \item  Linked libraries should match the dependencies of your module declared in the root otb-module.cmake file. 
\end{itemize}

The last line of CMake code takes care of installation instructions: 
\begin{verbatim}
otb_module_target(TBTheModuleName)
\end{verbatim}

The overall CMakeLists.txt file should look like: 

\begin{verbatim}
set(OTBTheModuleName_SRC
    sourceFile1.cxx
    sourceFile2.cxx
    sourceFile3.cxx
    ...
    sourceFileN.cxx)

add_library(OTBTheModuleName ${OTBTheModuleName_SRC})

target_link_libraries(OTBTheModuleName ${OTBModule1_LIBRARIES} ${OTBModule2_LIBRARIES} ... ${OTBModuleN_LIBRARIES})

otb_module_target(TBTheModuleName)
\end{verbatim}

\section{The app folder}

The app folder contains the code of applications shipped with your module. If your module has no application, you do not need the app folder.

\textbf{Notes}: If your module contains application (and an app folder), do not forget to add the ApplicationEngine in the dependencies listed in the otb-module.cmake file.

In addition to the applications source code, the app folder should contain a CMakeLists.txt file as follows.

For each application, a single call otb\textunderscore create\textunderscore application is required: 

\begin{verbatim}
otb_create_application(
 NAME           TheModuleApplication1
 SOURCES        TheModuleApplication1.cxx
 LINK_LIBRARIES ${OTBModule1_LIBRARIES} ${OTBModule2_LIBRARIES} ... ${OTBModuleN_LIBRARIES})
 
\end{verbatim}

\section{The test folder}

This folder contains tests of the module. If your module has no test in it (which is not recommended, you do not need it).

The test folder should contain the source files of tests, as well as a CMakeLists.txt file. This file will contain the following.

First, indicate that this folder contains tests. 

\begin{verbatim}
otb_module_test()
\end{verbatim}

Then, build the test driver: 

\begin{verbatim}
set(OTBTheModuleNameTests
    testFile1.cxx
    testFile2.cxx
    ...
    testFileN.cxx)

add_executable(otbTheModuleNameTestDriver ${OTBTheModuleNameTests})

target_link_libraries(otbTheModuleNameTestDriver ${OTBTheModuleName-Test_LIBRARIES})
 
otb_module_target_label(otbTheModuleNameTestDriver)
\end{verbatim}

Finally, you can add your tests: 

\begin{verbatim}
otb_add_test(NAME nameOfTheTest COMMAND otbTheModuleNameTestDriver
             --compare-image ${EPSILON_8} ... # baseline comparison if needed
             nameOfTheTestFunction
             testParameters)
\end{verbatim}

If your module contains one or more application in the app folder, you should
also write tests for them, in the test folder. Running an application test is
easily done with the helper macro otb\textunderscore test\textunderscore
application:

\begin{verbatim}
otb_test_application(NAME   nameofApplication1Test1
                      APP  TheModuleApplication1
                      OPTIONS -in1 ${INPUTDATA}/input1.tif
                              -in2 ${INPUTDATA}/input2.tif
                              -out ${TEMP}/nameofApplication1Test1_result.tif
                      VALID   --compare-image ${EPSILON_8}
                              ${BASELINE}/nameofApplication1Test1_result.tif
                              ${TEMP}/nameofApplication1Test1_result.tif)
\end{verbatim}

ToDo: Add instructions for test naming and input/baseline data inclusion.

You overall CMakeLists.txt file should look like: 

\begin{verbatim}
otb_module_test()

set(OTBTheModuleNameTests
    testFile1.cxx
    testFile2.cxx
    ...
    testFileN.cxx)

add_executable(otbTheModuleNameTestDriver ${OTBTheModuleNameTests})

target_link_libraries(otbTheModuleNameTestDriver ${OTBTheModuleName-Test_LIBRARIES})
 
otb_module_target_label(otbTheModuleNameTestDriver)

otb_add_test(NAME nameOfTheTest COMMAND otbTheModuleNameTestDriver
             --compare-image ${EPSILON_8} ... # baseline comparison if needed
             nameOfTheTestFunction
             testParameters)
\end{verbatim}

\section{Including a remote module in OTB}
\begin{itemize}
       \item Local build of a remote module  
\end{itemize}

Your remote module can be build inside the OTB source tree or outside as a
external CMake project with an existing OTB. Please note in that case
that you'll have to set OTB\textunderscore DIR CMake option.

If OTB\textunderscore DIR is an OTB build tree, there are two ways of compiling:
\begin{itemize}
  \item Build as a module, in which case build files will be written
    to the OTB build tree as other modules. Main benefit is that this
    will enrich the current OTB build with your new module, but you
    need to have write access to the build directory.
  \item Build as a standalone CMake project, in which case build files
    will remain in the module build folder. This build is fully
    independent from the build (or install) directory, but the module
    will not be recognized as an OTB module (still you will be able to
    use its binaries and libraries).
\end{itemize}

This behaviour is controlled by the OTB\textunderscore BUILD\textunderscore MODULE\textunderscore AS\textunderscore STANDALONE, which is OFF by default (hence first behaviour).

Note that when dealing with an installed OTB, only the second behaviour (build as standalone) is available.

Optionally, you can build your new remote module inside the OTB source tree by simply copy
the folder containing the module component to Modules/Remote, then run CMake
configuration. you should see a new CMake option named MODULE\textunderscore
TheModuleName. Simply turn this option to ON, and finish CMake
configuration. Your module will be built with the rest of the OTB project.

\begin{itemize}
       \item  Sharing your remote module 
\end{itemize}

To make your remote module available to others when building OTB, you should
provide a CMake file named TheModuleName.remote.cmake file for inclusion in the
Modules/Remote folder in OTB source tree.

This file should contain the following: 

\begin{verbatim}
#Contact: Author name <author email address>

otb_fetch_module(TheModuleName
  "A description of the module, to appear during CMake configuration step"
  GIT\textunderscore REPOSITORY http\textunderscore link\textunderscore to\textunderscore a\textunderscore git\textunderscore repository\textunderscore hosting\textunderscore the\textunderscore module
  GIT\textunderscore TAG the\textunderscore git\textunderscore revision\textunderscore to\textunderscore checkout
  )
\end{verbatim}
This file should be provided along with your remote module inclusion proposal email to the otb-developers list. Please refer to the contributors guidelines for more information (next section). 

