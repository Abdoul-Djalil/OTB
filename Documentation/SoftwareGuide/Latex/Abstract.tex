\chapter*{Foreword}
\noindent


Beside the Pleiades (PHR) and Cosmo-Skymed (CSK) systems developments
forming ORFEO, the dual and bilateral system (France - Italy) for
Earth Observation, the ORFEO Accompaniment Program was set up, to
prepare, accompany and promote the use and the exploitation of the
images derived from these sensors.

The creation of a preparatory
program\footnote{http://smsc.cnes.fr/PLEIADES/A\_prog\_accomp.htm} is
needed because of:
\begin{itemize}
\item the new capabilities and performances of the ORFEO systems
  (optical and radar high resolution, access capability, data quality,
  possibility to acquire simultaneously in optic and radar),
\item the implied need of new methodological developments : new
  processing methods, or adaptation of existing methods,
\item the need to realise those new developments in very close
  cooperation with the final users for better integration of new
  products in their systems.

\end{itemize}

This program was initiated by CNES mid-2003 and will last until mid
2013.  It consists in two parts, between which it is necessary to keep
a strong interaction:
\begin{itemize}
\item A Thematic part,
\item A Methodological part.
\end{itemize}

The Thematic part covers a large range of applications (civil and
defence), and aims at specifying and validating value added products
and services required by end users. This part includes consideration
about products integration in the operational systems or processing
chains. It also includes a careful thought on intermediary structures
to be developed to help non-autonomous users. Lastly, this part aims
at raising future users awareness, through practical demonstrations
and validations.

The Methodological part objective is the definition and the
development of tools for the operational exploitation of the
submetric optic and radar images (tridimensional aspects, changes
detection, texture analysis, pattern matching, optic radar
complementarities). It is mainly based on R\&D studies and doctorate
and post-doctorate researches.

In this context, CNES\footnote{http://www.cnes.fr} decided to develop
the \emph{ORFEO ToolBox} (OTB), a set of algorithms encapsulated in a
software library. The goals of the OTB is to capitalise a methological
\textit{savoir faire} in order to adopt an incremental development
approach aiming to efficiently exploit the results obtained in the
frame of methodological R\&D studies.

All the developments are based on FLOSS (Free/Libre Open Source
Software) or existing CNES developments. OTB is distributed under the
C\'eCILL licence,
\url{http://www.cecill.info/licences/Licence_CeCILL_V2-en.html}.

OTB is implemented in C++ and is mainly based on
ITK\footnote{http://www.itk.org} (Insight Toolkit).


%% L'environnement de l'OTB est mis en place par l'outil CMake\footnote{http://www.cmake.org},
%% permettant ainsi de g\'{e}rer les proc\'{e}dures de compilation, g\'{e}n\'{e}ration et d'installation et ce quelque sois la plate forme cible.

%% Dans un souci d'homog\'{e}n\'{e}isation, l'OTB est con\c{c}ue et d\'{e}velopp\'{e}e suivant la philosophie et les principes \'{e}dict\'{e}s
%% par la biblioth\`{e}que ITK (programmation g\'{e}n\'{e}rique, m\'{e}canisme des \emph{Object Factories}, \emph{Smart pointers}, exceptions, \emph{Multi-Threading}, etc...).
%% Ces principes sont pr\'{e}sent\'{e}s dans le paragraphe \emph{3.2 Essential System Concepts} du guide ITK \url{http://www.itk.org/ItkSoftwareGuide.pdf}

%% Enfin, la m\'{e}thodologie de d\'{e}veloppement appliqu\'{e}e s'appuie sur une approche it\'{e}rative bas\'{e}e sur la programmation agile :
%% le sch\'{e}ma de d\'{e}veloppement suit le cycle \'{e}dict\'{e}e par la m\'{e}thodolgie de l'eXtreme Programming (XP)\footnote{http://www.xprogramming.com}.



%% Ce document constitue le guide d'utilisation et de d\'{e}veloppement de l'OTB. La version la plus r\'{e}cente de ce document est accessible \`{a}
%% \url{http://smsc.cnes.fr/PLEIADES/Fr/A_prog_accomp.htm/OTB/otbSoftwareGuide.pdf}.


