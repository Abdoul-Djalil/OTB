\chapter{Dimension Reduction}\label{chap:dimred}

Dimension reduction is a statistical process, which concentrates the
amount of information in multivariate data into a fewer number of
variables (or dimensions). An interesting review of the domain has been done by Fodor~\cite{Fodor2002dimensionred}.

Though there are plenty of non-linear methods in the litterature, OTB
provides only linear dimension reduction techniques applied to images for now.

Usually, linear dimension-reduction algorithms try to find a set of
linear combinations of the input image bands that maximise a given
criterion, often chosen so that image information concentrates on the
first components. Algorithms differs by the criterion to optimise and
also by their handling of the signal or image noise.

In remote-sensing images processing, dimension reduction algorithms
are of great interest for denoising, or as a preliminary processing
for classification of feature images or unmixing of hyperspectral
images. In addition to the denoising effect, the advantage of
dimension reduction in the two latter is that it lowers the size of
the data to be analysed, and as such, speeds up the processing time
without too much loss of accuracy.

\section{Principal Component Analysis}

\input{PCAExample}

\section{Noise-Adjusted Principal Components Analysis}

\input{NAPCAExample}

\section{Maximum Noise Fraction}

\input{MNFExample}

\section{Fast Independant Component Analysis}

\input{ICAExample}

\section{Maximum Autocorrelation Factor}

\input{MaximumAutocorrelationFactor}


