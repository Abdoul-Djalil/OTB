\chapter{Change Detection}
\section{Introduction}
Change detection techniques try to detect and locate areas which have
changed between two or more observations of the same scene. These
changes can be of different types, with different origins and of
different temporal length. This allows to distinguish different kinds
of applications:
\begin{itemize}
\item \emph{land use monitoring}, which corresponds to the
  characterization of the evolution of the vegetation, or its seasonal
  changes;
\item \emph{natural resources management}, which corresponds mainly
  to the characterization of the evolution of the urban areas, the
  evolution of the deforestation, etc.
\item \emph{damage mapping}, which corresponds to the location of
  damages caused by natural or industrial disasters.
\end{itemize}

From the point of view of the observed phenomena, one can distinguish
2 types of changes whose nature is rather different: the abrupt
changes and the progressive changes, which can eventually be
periodic. From the data point of view, one can have:

       \begin{itemize}
       \item Image pairs before and after the event. The applications
       are mainly the abrupt changes.

	 \item Multi-temporal image series on which 2 types on changes
	 may appear:
	 \begin{itemize}
	   \item The slow changes like for instance the erosion,
	   vegetation evolution, etc. The knowledge of the studied
	   phenomena and of their consequences on the geometrical
	   and radiometrical evolution at the different dates is a
	   very important information for this kind of analysis.

	     \item The abrupt changes may pose different kinds of
	     problems depending on whether the date of the change is
	     known in the image series or not. The detection of areas
	     affected by a change occurred at a known date may exploit
	     this a priori information in order to split the image
	     series into two sub-series (before an after) and use the
	     temporal redundancy in order to improve the detection
	     results. On the other hand, when the date of the change
	     is not known, the problem has a higher difficulty.

	 \end{itemize}
	 
       \end{itemize}

From this classification of the different types of problems, one can
infer 4 cases for which one can look for algorithms as a function of
the available data:
\begin{enumerate}
\item Abrupt changes in an image pair. This is no doubt the field for
  which more work has been done. One can find tools at the 3 classical
  levels of image processing: data level (differences, ratios, with
  or without pre-filtering, etc.), feature level (edges, targets,
  etc.), and interpretation level (post-classification comparison).

\item Abrupt changes within an image series and a known date. One can
  rely on bi-date techniques, either by fusing the images into 2 stacks
  (before and after), or by fusing the results obtained by different
  image couples (one after and one before the event). One can also use
  specific discontinuity detection techniques to be applied in the
  temporal axis.

\item Abrupt changes within an image series and an unknown date. This
  case can be seen either as a generalization of the preceding one (testing
  the N-1 positions for N dates) or as a particular case of the
  following one.

\item Progressive changes within an image series. One can work in two
  steps:
  \begin{enumerate}
    \item detect the change areas using stability criteria in the
    temporal areas;
    \item identify the changes using prior information about the type
    of changes of interest.
  \end{enumerate}
  
\end{enumerate}




\subsection{Surface-based approaches}\label{secChgtAbr}
In this section we discuss about the damage assessment techniques
which can be applied when only two images (before/after) are available.\\

As it has been shown in recent review works
\cite{Coppin03,Lu04,Radke05,Richards05}, a relatively high number of
methods exist, but most of them have been developed for optical and
infrared sensors. Only a few recent works on change detection with
radar images exist
\cite{Stabel02,Bruzzone02b,Onana_2003,Inglada03,Derrode03,Bazi05,Inglada07}.
However, the intrinsic limits of passive sensors, mainly related to
their dependence on meteorological and illumination conditions, impose
severe constraints for operational applications. The principal
difficulties related to change detection are of four types:

\begin{enumerate}
\item In the case of radar images, the speckle noise makes the image
  exploitation difficult.
\item The geometric configuration of the image acquisition can produce
  images which are difficult to compare.
\item Also, the temporal gap between the two acquisitions and thus the
  sensor aging and the inter-calibration are sources of variability
  which are difficult to deal with.
\item Finally, the normal evolution of the observed scenes must not be
  confused with the changes of interest.
\end{enumerate}

The problem of detecting abrupt changes between a pair of images is
the following: Let $I_{1},I_{2}$ be two images acquired at different
dates $t_{1},t_{2}$; we aim at producing a thematic map which shows
the areas where changes have taken place.

Three main categories of methods exist:

\begin{itemize}
\item{Strategy $1$: Post Classification Comparison}

The principle of this approach \cite{Deer_1998} is two obtain two
land-use maps independently for each date and comparing them. 



\item{Strategy $2$: Joint classification}

This method consists in producing the change map directly from a joint
classification of both images.

\item{Strategy $3$: Simple detectors}

The last approach consists in producing an image of change likelihood
(by differences, ratios or any other approach) and thresholding it in
order to produce the change map.

\end{itemize}


Because of its simplicity and its low computation overhead, the third
strategy is the one which has been chosen for the processing
presented here.



\section{Change Detection Framework}
\label{sec:ChangeDetectionFramework}
\input{ChangeDetectionFrameworkExample.tex}
\section{Simple Detectors}
\label{sec:SimpleDetectors}
\subsection{Mean Difference}
\label{sec:MeanDifference}

The simplest change detector is based on the pixel-wise differencing
of image values: 
\begin{equation}
I_{D}(i,j)=I_{2}(i,j)-I_{1}(i,j).
\end{equation}

In order to make the algorithm robust to noise, one actually uses
local means instead of pixel values.

\input{DiffChDet}

\subsection{Ratio Of Means}
\label{sec:RatioOfMeans}

This detector is similar to the previous one except that it uses a
ratio instead of the difference:
\begin{equation}
\displaystyle I_{R}(i,j) = \frac{\displaystyle I_{2}(i,j)}{\displaystyle I_{1}(i,j)}.
\end{equation}

The use of the ratio makes this detector robust to multiplicative
noise which is a good model for the speckle phenomenon which is
present in radar images.

In order to have a bounded and normalized detector the following
expression is actually used:


\begin{equation}
\displaystyle I_{R}(i,j) = 1 - min \left(\frac{\displaystyle I_{2}(i,j)}{\displaystyle I_{1}(i,j)},\frac{\displaystyle I_{1}(i,j)}{\displaystyle I_{2}(i,j)}\right).
\end{equation}


\input{RatioChDet}


\section{Statistical Detectors}
\label{sec:StatisticalDetectors}

\subsection{Distance between local distributions}
\label{sec:KullbackLeiblerDistance}

This detector is similar to the ratio of means detector (seen in the 
previous section page~\pageref{sec:RatioOfMeans}). Nevertheless, 
instead of the comparison of means, the comparison is performed to
the complete distribution of the two Random Variables (RVs)~\cite{Inglada03}.

The detector is based on the Kullback-Leibler distance between probability 
density functions (pdfs). In the neighborhood of each pixel of the pair 
of images $I_1$ and $I_2$ to be compared, the distance between local pdfs 
$f_1$ and $f_2$ of RVs $X_1$ and $X_2$ is evaluated by:
\begin{align}
  {\cal K}(X_1,X_2) &= K(X_1|X_2) + K(X_2|X_1) \\
  \text{with} \qquad
  K(X_j | X_i) &= \int_{\mathbbm{R}} 
      \log \frac{f_{X_i}(x)}{f_{X_j}(x)} f_{X_i}(x) dx,\qquad i,j=1,2.
\end{align}
In order to reduce the computational time, the local pdfs $f_1$ and $f_2$ 
are not estimated through histogram computations but rather by a cumulant
expansion, namely the Edgeworth expansion, with is based on the 
cumulants of the RVs:
\begin{equation}\label{eqEdgeworthExpansion}
f_X(x) = \left( 1 + \frac{\kappa_{X;3}}{6} H_3(x) 
					+ \frac{\kappa_{X;4}}{24} H_4(x)
					+ \frac{\kappa_{X;5}}{120} H_5(x)
					+ \frac{\kappa_{X;6}+10 \kappa_{X;3}^2}{720} H_6(x) \right) {\cal G}_X(x).
\end{equation}
In eq.~\eqref{eqEdgeworthExpansion}, ${\cal G}_X$ stands for the Gaussian pdf
which has the same mean and variance as the RV $X$. The $\kappa_{X;k}$
coefficients are the cumulants of order $k$, and $H_k(x)$ are the 
Chebyshev-Hermite polynomials of order $k$ (see~\cite{Inglada07} for deeper
explanations).

\input{KullbackLeiblerDistanceChDet}

\subsection{Local Correlation}
\label{sec:LocalCorrelation}
The correlation coefficient measures the likelihood of a linear
relationship between two random variables:
\begin{equation}
\begin{split}
I_\rho(i,j) &= \frac{1}{N}\frac{\sum_{i,j}(I_1(i,j)-m_{I_1})(I_2(i,j)-m_{I_2})}{\sigma_{I_1}
\sigma_{I_2}}\\
& = \sum_{(I_1(i,j),I_2(i,j))}\frac{(I_1(i,j)-m_{I_1})(I_2(i,j)-m_{I_2})}{\sigma_{I_1}
\sigma_{I_2}}p_{ij}
\end{split}
\end{equation}

where $I_1(i,j)$ and $I_2(i,j)$ are the pixel values of the 2 images and
$p_{ij}$ is the joint probability density. This is like using a linear model:
\begin{equation}
I_2(i,j) = (I_1(i,j)-m_{I_1})\frac{\sigma_{I_2}}{\sigma_{I_1}}+m_{I_2}
\end{equation}
for which we evaluate the likelihood with  $p_{ij}$.

With respect to the difference detector, this one will be robust to
illumination changes.
\input{CorrelChDet.tex}

%% \subsection{Mutual Information}

%% Other sophisticated change detectors can be used by applying some
%% concepts of information theory. We have chosen to implement several
%% detectors based on the mutual information measure
%% \cite{Thevenaz2000,Inglada_2002}. This kind of measure needs for the
%% estimation of the joint density probabilities for the pair of images
%% to be compared. Depending on how this estimation is made, one can
%% choose between robust but slow detectors or quick but less robust ones.\\

%% The mutual information is a divergence (some kind of distance) between
%% the joint probability $p_{1,2}$ and the product of marginal ones
%% $p_1\cdot p_2$. Therefore, it is a measure of statistical dependence
%% between the two images and can thus be understood as a generalization
%% of the correlation coefficient. This means that it can be applied to
%% the multi-sensor case.\\

%% The divergence used is written as:
%% \begin{equation}
%% K(P,Q) = \int p \log\frac{p}{q},
%% \end{equation}

%% so the mutual information detector is written as:


%% \begin{equation}
%% I_{MI}(i,j) = \int p_{1,2} \log\frac{p_{1,2}}{p_1\cdot p_2}.
%% \end{equation}


%% \subsubsection{Joint histogram}
%% In this version of the detector, a joint probability density $p_{ij}$ is
%% estimated only once for the pair of images. This makes it a rather
%% quick detector.
%% \subsubsection{Local histogram}
%% This version uses a local estimation of the probabilities in the
%% neighborhood of each pixel. It is the slowest detector, but the most
%% robust one.
%% \subsubsection{Cumulant-based}

%% This version is the quickest one, but it is only an approximation of the
%%     mutual information. Indeed, a probability density can be
%%     reconstructed from a series expansion of its cumulants. The
%%     cumulants are defined as follows:
%% \begin{subequations}
%% \begin{equation}
%% E\left[\prod_{k \in N} X_k\right]=\sum_{N_1\cup\cdot\cdot\cdot \cup
%% N_n=N}cum(X_k, k \in N_1)\cdot\cdot\cdot cum(X_k,k\in N_n)=\kappa_k,
%% \end{equation}
%% \begin{equation}
%% cum(X_k, k\in N)=\sum_{N_1\cup\cdot\cdot\cdot \cup N_n=N}
%% (-1)^{n-1}(n-1)!E\left[\prod_{k\in N_1} X_k \right]\cdot\cdot\cdot
%% E\left[\prod_{k\in N_n} X_k \right],
%% \end{equation}
%% \end{subequations}

%% For instance, one has 
        
%% \begin{equation}
%% cum(X_1,X_2)=E(X_1,X_2)-(EX_1)(EX_2)=cov(X_1,X_2).
%% \end{equation}

%% \begin{equation}
%%   \begin{split}
%%     cum(X_1,X_2,X_3)=&E(X_1,X_2,X_3)-E(X_1,X_2)(EX_3)-E(X_1,X_3)(EX_2)\\
%%     & -E(X_2,X_3)(EX_1)+2(EX_1)(EX_2)(EX_3)
%%   \end{split}
%% \end{equation}

%% Using these cumulants, the series expansion of the probability density
%% function $f(x)$ can be written as a modulation of the normalized
%% Gaussian function $\Phi(x)$:


%%       \begin{equation}
%%   f(x) \approx  \Phi(x)\left[ P_0(x) +
%%   P_1(x)\frac{1}{\sqrt{n}}+ P_2(x)\frac{1}{n} + ...+ P_r(x)\frac{1}{n^{r/2}}\right],
%% \end{equation}
%% with
%% \begin{subequations}
%%   \begin{equation}
%%     P_0(x) = 1,
%%   \end{equation}
%%   \begin{equation}
%%     P_1(x) = \frac{\kappa_3}{3!}H_3(x),
%%   \end{equation}
%%   \begin{equation}
%%     P_2(x) = \frac{\kappa_4}{4!}H_4(x) + \frac{10\kappa_3^2}{6!}H_6(x),
%%   \end{equation}
%% \end{subequations}
%% and the Hermite polynomials
%% \begin{subequations}
%%   \begin{equation}
%%     H_0(x) = 1,
%%   \end{equation}
%%   \begin{equation}
%%     H_1(x) = x,
%%   \end{equation}
%%   \begin{equation}
%%     H_2(x) = x^2 -1,
%%   \end{equation}
%%   \begin{equation}
%%     H_3(x) = x^3-3x.
%%   \end{equation}
%% \end{subequations}

%% When thiese approximations are used in the expression of the mutual
%% information, one has the following result:
    
%%     \begin{equation}
%% I_{IM}(i,j)({\underline Y})\approx \frac{1}{4}\sum_{kl\neq
%% kk}\left(cum_2(Y_k,Y_l)\right)^2+\frac{1}{48}\sum_{klmn\neq
%% kkkk}\left(cum_4(Y_k,Y_l,Y_m,Y_n)\right)^2,
%% \label{kim}
%% \end{equation}
%%  where $\{k,l,m,n\}$ can take the values 1 and 2 (the image index) and
%%  the cumulants are computed in the neighborhood if the pixel of
%%  coordinates $(i,j)$.


\section{Multi-Scale Detectors}
\label{sec:MultiScaleDetectors}

\subsection{Kullback-Leibler Distance between distributions}
\label{sec:KullbackLeiblerProfile}

This technique is an extension of the distance between distributions 
change detector presented in section~\ref{sec:KullbackLeiblerDistance}.
Since this kind of detector is based on cumulants estimations through
a sliding window, the idea is just to upgrade the estimation of the cumulants
by considering new samples as soon as the sliding window is increasing in size.

Let's consider the following problem: how to update the moments when a
$N+1^{th}$ observation $x_{N+1}$ is added to a set of observations $\{x_1, x_2, \ldots,
x_N\}$ already considered.
The evolution of the central moments may be characterized by:
\begin{align}\label{eqMomentN}
	\mu_{1,[N]} & = \frac{1}{N} s_{1,[N]} \\
	\mu_{r,[N]} & = \frac{1}{N} \sum_{\ell = 0}^r \binom{r}{\ell} 
									\left( -\mu_{1,[N]} \right)^{r-\ell}
									s_{\ell,[N]}, \notag
\end{align}
where the
notation $s_{r,[N]} = \sum_{i=1}^N x_i^r$ has been used.
Then, Edgeworth series is updated also by transforming moments to
cumulants by using:
\begin{equation}\label{eqCumsMoms}
  \begin{split}
\kappa_{X;1} &= \mu_{X;1}\\
\kappa_{X;2} &= \mu_{X;2}-\mu_{X;1}^2\\
\kappa_{X;3} &= \mu_{X;3} - 3\mu_{X;2} \mu_{X;1} + 2\mu_{X;1}^3\\
\kappa_{X;4} &= \mu_{X;4} - 4\mu_{X;3} \mu_{X;1} - 3\mu_{X;2}^2 + 12 \mu_{X;2} \mu_{X;1}^2 - 6\mu_{X;1}^4.
  \end{split}
\end{equation}
It yields a set of images that represent the change measure according to an
increasing size of the analysis window.

\input{KullbackLeiblerProfileChDet}

\section{Multi-components detectors}

\subsection{Multivariate Alteration Detector}

\input{MultivariateAlterationDetector}



