\chapter{Radiometry}

Remote sensing is not just a matter of taking pictures, but also --
mostly -- a matter of measuring physical values. In order to properly
deal with physical magnitudes, the numerical values provided by the
sensors have to be calibrated. After that, several indices with
physical meaning can be computed.

%%%%%%%%%%%%%%%%%%%%%%%%%%%%%%%%%%%%%%%%%%%%%%%%%%%%%%%%%%%%%%%%%%%%%%
\section{Radiometric Indices}
\label{sec:VegetationIndex}
\label{sec:RadiometricIndex}

\subsection{Introduction}

With multispectral sensors, several indices can be computed, combining several
spectral bands to show features that are not obvious using only one band.
Indices can show:
\begin{itemize}
  \item Vegetation (Tab~\ref{tab:vegetationindices})
  \item Soil (Tab~\ref{tab:soilindices})
  \item Water (Tab~\ref{tab:waterindices})
  \item Built up areas (Tab~\ref{tab:builtupindices})
\end{itemize}

A vegetation index is a quantitative measure used to measure biomass
or vegetative vigor, usually formed from combinations of several
spectral bands, whose values are added, divided, or multiplied in
order to yield a single value that indicates the amount or vigor of
vegetation.

Numerous indices are available in OTB and are listed in
table~\ref{tab:vegetationindices} to \ref{tab:builtupindices} with their
references.

\begin{table}[htb]
\centering
\begin{tabular}{|c|l|}
\hline
NDVI &  Normalized Difference Vegetation Index \cite{Rouse1973-NDVI} \\
RVI &  Ratio Vegetation Index \cite{Pearson1972-RVI}\\
PVI & Perpendicular Vegetation Index \cite{Richardson1977-PVI,Wiegand1991-PVI}\\ 
SAVI & Soil Adjusted Vegetation Index \cite{Huete1988-SAVI} \\
TSAVI & Transformed Soil Adjusted Vegetation Index \cite{Baret1989-TSAVI,Baret1991-TSAVI} \\
MSAVI & Modified Soil Adjusted Vegetation Index  \cite{Qi1994-MSAVI} \\
MSAVI2 & Modified Soil Adjusted Vegetation Index  \cite{Qi1994-MSAVI} \\
GEMI &   Global Environment Monitoring Index \cite{Pinty1992-GEMI} \\
WDVI & Weighted Difference Vegetation Index  \cite{Clevers1988-WDVI,Clevers1991-WDVI} \\ 
AVI & Angular Vegetation Index  \cite{AVI}\\
ARVI & Atmospherically Resistant  Vegetation Index \cite{ARVI} \\
TSARVI & Transformed Soil Adjusted Vegetation Index   \cite{ARVI} \\
EVI & Enhanced Vegetation Index \cite{Huete1994-EVI,Justice1998-EVI} \\
IPVI & Infrared Percentage Vegetation Index  \cite{Crippen1990-IPVI} \\
TNDVI & Transformed NDVI  \cite{Deering1975-TNDVI} \\
\hline
\end{tabular}
\caption{Vegetation indices}\label{tab:vegetationindices}
\end{table}

\begin{table}[htb]
\centering
\begin{tabular}{|c|l|}
\hline
IR  & Redness Index  \cite{Pouget1990-IRIC} \\
IC  & Color Index  \cite{Pouget1990-IRIC} \\
IB  & Brilliance Index  \cite{Nicoloyanni1990-IB} \\
IB2 & Brilliance Index  \cite{Nicoloyanni1990-IB} \\
\hline
\end{tabular}
\caption{Soil indices}\label{tab:soilindices}
\end{table}

\begin{table}[htb]
\centering
\begin{tabular}{|c|l|}
\hline
SRWI & Simple Ratio Water Index \cite{ZarcoTejada2001-SRWI} \\
NDWI & Normalized Difference Water Index  \cite{Gao1996-NDWI} \\
NDWI2 &  Normalized Difference Water Index \cite{McFeeters1996-NDWI2} \\
MNDWI &  Modified Normalized Difference Water Index  \cite{Xu2006-MNDWI} \\
NDPI &  Normalized Difference Pond Index \cite{Lacaux2007-NDTI} \\
NDTI &  Normalized Difference Turbidity Index  \cite{Lacaux2007-NDTI} \\
SA & Spectral Angle \\
\hline
\end{tabular}
\caption{Water indices}\label{tab:waterindices}
\end{table}

\begin{table}[htb]
\centering
\begin{tabular}{|c|l|}
\hline
NDBI &  Normalized Difference Built Up Index \cite{Zha2003-NDBI} \\
ISU &  Index Surfaces Built \cite{Abdellaoui1997-ISU} \\
\hline
\end{tabular}
\caption{Built-up indices}\label{tab:builtupindices}
\end{table}


The use of the different indices is very similar, and only few example are
given in the next sections.

\subsection{NDVI}
\label{secNDVI}
NDVI was one of the most successful of many attempts to simply and
quickly identify vegetated areas and their {\em condition}, and it remains
the most well-known and used index to detect live green plant canopies
in multispectral remote sensing data. Once the feasibility to detect
vegetation had been demonstrated, users tended to also use the NDVI to
quantify the photosynthetic capacity of plant canopies. This, however,
can be a rather more complex undertaking if not done properly.
\input{NDVIRAndNIRVegetationIndexImageFilter.tex}

\subsection{ARVI}
\label{secARVI}
\input{ARVIMultiChannelRAndBAndNIRVegetationIndexImageFilter.tex}

\subsection{AVI}
\label{secAVI}
\input{AVIMultiChannelRAndGAndNIRVegetationIndexImageFilter.tex}




\section{Atmospheric Corrections}
\label{secAtmosphericCorrections}
\input{AtmosphericCorrectionSequencement.tex}
