\chapter{Image Segmentation}
Segmentation of remote sensing images is a challenging task. A myriad
of different methods have been proposed and implemented in recent
years. In spite of the huge effort invested in this problem, there is
no single approach that can generally solve the problem of
segmentation for the large variety of image modalities existing today.

The most effective segmentation algorithms are obtained by carefully
customizing combinations of components. The parameters of these components are
tuned for the characteristics of the image modality used as input and the
features of the objects to be segmented. 

The Insight Toolkit provides a basic set of algorithms that can be used to
develop and customize a full segmentation application. They are
therefore available in the Orfeo Toolbox. Some of the most
commonly used segmentation components are described in the following
sections.


\section{Region Growing}

Region growing algorithms have proven to be an effective approach for image
segmentation. The basic approach of a region growing algorithm is to start
from a seed region (typically one or more pixels) that are considered to be
inside the object to be segmented. The pixels neighboring this region are
evaluated to determine if they should also be considered part of the
object. If so, they are added to the region and the process continues as long
as new pixels are added to the region.  Region growing algorithms vary
depending on the criteria used to decide whether a pixel should be included
in the region or not, the type connectivity used to determine neighbors, and
the strategy used to visit neighboring pixels.

Several implementations of region growing are available in ITK.  This section
describes some of the most commonly used.

\subsection{Connected Threshold}

A simple criterion for including pixels in a growing region is to evaluate
intensity value inside a specific interval.

\label{sec:ConnectedThreshold}
\input{ConnectedThresholdImageFilter.tex}

\subsection{Otsu Segmentation}
Another criterion for classifying pixels is to minimize the error of misclassification.
The goal is to find a threshold that classifies the image into two clusters such that 
we minimize the area under the histogram for one cluster that lies on the other cluster's 
side of the threshold. This is equivalent to minimizing the within class variance
or equivalently maximizing the between class variance.

\label{sec:OtsuThreshold}
\ifitkFullVersion 
\input{OtsuThresholdImageFilter.tex}
\fi

\label{sec:OtsuMultipleThreshold}
\ifitkFullVersion 
\input{OtsuMultipleThresholdImageFilter.tex}
\fi

\subsection{Neighborhood Connected}
\label{sec:NeighborhoodConnectedImageFilter}
\ifitkFullVersion 
\input{NeighborhoodConnectedImageFilter.tex}
\fi


\subsection{Confidence Connected}
\label{sec:ConfidenceConnected}
\ifitkFullVersion 
\input{ConfidenceConnected.tex}
%\input{ConfidenceConnectedOnBrainWeb.tex}
\fi




%\subsection{Isolated Connected}
%\label{sec:IsolatedConnected}
%\ifitkFullVersion 
%\input{IsolatedConnectedImageFilter.tex}
%\fi


%\subsection{Confidence Connected in Vector Images}
%\label{sec:VectorConfidenceConnected}
%\ifitkFullVersion 
%\input{VectorConfidenceConnected.tex}
%\fi


\section{Segmentation Based on Watersheds}
\label{sec:WatershedSegmentation}
\ifitkFullVersion 
%
%
%  This file is inserted in the file Segmentation.tex
%
%

\subsection{Overview}
\label{sec:AboutWatersheds}
\index{Watersheds}
\index{Watersheds!Overview}
Watershed segmentation classifies pixels into regions using gradient descent on
image features and analysis of weak points along region boundaries.  Imagine
water raining onto a landscape topology and flowing with gravity to collect in
low basins.  The size of those basins will grow with increasing amounts of
precipitation until they spill into one another, causing small basins to merge
together into larger basins.  Regions (catchment basins) are formed by using
local geometric structure to associate points in the image domain with local
extrema in some feature measurement such as curvature or gradient magnitude.
This technique is less sensitive to user-defined thresholds than classic
region-growing methods, and may be better suited for fusing different types of
features from different data sets.  The watersheds technique is also more
flexible in that it does not produce a single image segmentation, but rather a
hierarchy of segmentations from which a single region or set of regions can be
extracted a-priori, using a threshold, or interactively, with the help of a
graphical user interface
\cite{Yoo1992,Yoo1991}.

The strategy of watershed segmentation is to treat an image $f$ as a height
function, i.e.,  the surface formed by graphing $f$ as a function of its
independent parameters, $\vec{x} \in U$.  The image $f$ is often not the
original input data, but is derived from that data through some filtering,
graded (or fuzzy) feature extraction, or fusion of feature maps from different
sources.  The assumption is that higher values of $f$ (or $-f$) indicate the
presence of boundaries in the original data.  Watersheds may therefore be
considered as a final or intermediate step in a hybrid segmentation method,
where the initial segmentation is the generation of the edge feature map.

Gradient descent associates regions with local minima of $f$ (clearly interior
points) using the watersheds of the graph of $f$, as in
Figure~\ref{fig:segment}.
\begin{figure}
\centering
\includegraphics[width=0.9\textwidth]{WatershedCatchmentBasins.eps}
\itkcaption[Watershed Catchment Basins]{A fuzzy-valued boundary map, from an image
  or set of images, is segmented using local minima and catchment basins.}
\protect\label{fig:segment}
\end{figure}
That is, a segment consists of all points in $U$ whose paths of steepest
descent on the graph of $f$ terminate at the same minimum in $f$.  Thus, there
are as many segments in an image as there are minima in $f$.  The segment
boundaries are ``ridges'' \cite{Koenderink1979,Koenderink1993,Eberly1996} in
the graph of $f$.  In the 1D case ($U \subset \Re$), the watershed boundaries
are the local maxima of $f$, and the results of the watershed segmentation is
trivial.  For higher-dimensional image domains, the watershed boundaries are
not simply local phenomena; they depend on the shape of the entire watershed.

The drawback of watershed segmentation is that it produces a region for each
local minimum---in practice too many regions---and an over segmentation
results.  To alleviate this, we can establish a minimum watershed depth.  The
watershed depth is the difference in height between the watershed minimum and
the lowest boundary point.  In other words, it is the maximum depth of water
a region could hold without flowing into any of its neighbors.  Thus, a
watershed segmentation algorithm can sequentially combine watersheds whose
depths fall below the minimum until all of the watersheds are of sufficient
depth.  This depth measurement can be combined with other saliency
measurements, such as size.  The result is a segmentation containing regions
whose boundaries and size are significant.  Because the merging process is
sequential, it produces a hierarchy of regions, as shown in
Figure~\ref{fig:watersheds}.
\begin{figure}
\centering
\includegraphics[width=0.9\textwidth]{WatershedsHierarchy.eps}
\itkcaption[Watersheds Hierarchy of Regions]{A watershed segmentation combined
with a saliency measure (watershed depth) produces a hierarchy of regions.
Structures can be derived from images by either thresholding the saliency
measure or combining subtrees within the hierarchy.}
\protect\label{fig:watersheds}
\end{figure}
Previous work has shown the benefit of a user-assisted approach that provides
a graphical interface to this hierarchy, so that a technician can quickly move
from the small regions that lie within an area of interest to the union of
regions that correspond to the anatomical structure \cite{Yoo1991}.

There are two different algorithms commonly used to implement watersheds:
top-down and bottom-up.  The top-down, gradient descent strategy was chosen for
ITK because we want to consider the output of multi-scale differential
operators, and the $f$ in question will therefore have floating point
values. The bottom-up strategy starts with seeds at the local minima in the
image and grows regions outward and upward at discrete intensity levels
(equivalent to a sequence of morphological operations and sometimes called {\em
morphological watersheds} \cite{Serra1982}.) This limits the accuracy by
enforcing a set of discrete gray levels on the image.

\begin{figure}
\centering
\includegraphics[width=0.9\textwidth]{WatershedImageFilter.eps}
\itkcaption[Watersheds filter composition]{The construction
of the Insight watersheds filter.}
\protect\label{fig:constructionWatersheds}
\end{figure}

Figure~\ref{fig:constructionWatersheds} shows how the ITK image-to-image
watersheds filter is constructed.  The filter is actually a collection of
smaller filters that modularize the several steps of the algorithm in a
mini-pipeline.  The segmenter object creates the initial segmentation via
steepest descent from each pixel to local minima. Shallow background regions
are removed (flattened) before segmentation using a simple minimum value
threshold (this helps to minimize oversegmentation of the image).  The
initial segmentation is passed to a second sub-filter that generates a
hierarchy of basins to a user-specified maximum watershed depth.  The
relabeler object at the end of the mini-pipeline uses the hierarchy and the
initial segmentation to produce an output image at any scale {\em below} the
user-specified maximum.  Data objects are cached in the mini-pipeline so that
changing watershed depths only requires a (fast) relabeling of the basic
segmentation.  The three parameters that control the filter are shown in
Figure~\ref{fig:constructionWatersheds} connected to their relevant
processing stages.

\subsection{Using the ITK Watershed Filter}
\label{sec:UsingWatersheds}
\index{Watersheds!ImageFilter}
\input{WatershedSegmentation.tex}

%\subsection{Interpreting the Results}
%\label{sec:VisualizingWatersheds}
%\index{Watersheds!Visualization}
%In order to interpret the output of the Insight watersheds algorithm, it is
%important to understand what the output represents and how it is formatted. The
%itk::WatershedImageFilter produces an image of unsigned long integers.  Each
%integer number is a label for a unique segmented region (catchment basin) from
%the original input.  The output is the same size and dimensionality of the
%input.

%Because the segmented image may have potentially many thousands of labels, some
%care must be taken when visualizing the data or information may be lost.  One
%effective way to visualize the output is to map the integer labels into
%distinct RGB colors.  Because labels close in value tend to also be close
%spatially in the image, it is helpful to spread sequential label values far
%apart in the RGB range.  A hashing scheme that puts more weight on the
%least-significant integer bits is a good way to accomplish this.
%Figure~\ref{fig:colorVisWatersheds} shows a slice taken from a segmentation of
%a section of abdomen from the Visible Female Cryosection data.  The unsigned
%long label values of the output have been hashed into RGB colors.

%\begin{figure}
%\centering
%\includegraphics[width=.95\textwidth]{WatershedAbdomenSegmentation.eps}
%\itkcaption[Watershed segmentation of visible woman data]{A slice from a
%segmentation of Visible Female cryosection data.
%The original is shown at the left and the segmented image is shown to the
%right. Colored regions in the segmented image correspond to structures in the
%original data. }
%\protect\label{fig:colorVisWatersheds}
%\end{figure}

%For volumetric data, it is often interesting to create a surface rendering of
%one or more regions in the output.  This can be done by thresholding the
%region(s) of interest from the output image and exporting the result to a
%visualization package capable of isosurface rendering.  Thresholding can be
%done either by explicit manipulation of the image values through an ITK image
%iterator, or using one of the several Insight image thresholding filters.

%Figure~\ref{fig:surfaceRenderingWatersheds} is a surface rendering of the right
%eye, the optic nerve and chiasm, the lens of the eye, and the right lateral
%rectus muscle.  A slice from the original Visible Female head and neck
%cryosection data from which the segmentations were created is shown at the
%left.  This image was created as described above by thresholding isovalues in a
%watershed segmentation output and then rendered using third-party visualization
%software.

%\begin{figure}
%\centering
%\includegraphics[width=0.95\textwidth]{WatershedRendering.eps}
%\itkcaption[Watershed segmenation visualization]{A surface rendering (right)
%of four anatomical structures in the Visible
%Female head and neck. A slice of the data from which the segmentation was
%created is shown at the left.  The right and left optic nerves and chiasm are
%shown in yellow.  The right eye is in transparent purple.  The lens is dark
%purple.  The structure in red is the lateral rectus muscle.}
%\protect\label{fig:surfaceRenderingWatersheds}
%\end{figure}


\fi


% the clearpage command helps to avoid orphans in the title of the next
% section.
\clearpage

\section{Level Set Segmentation}
\label{sec:LevelSetsSegmentation}
\ifitkFullVersion 
%%%%%%%%%%%%%%%%%%%%%%%%%%%%%%%%%%%%%%%%%%%%%%%%%%%%%%%%%%%%%%%%%%%%%%%%
%
%
%     This file is included from the file   Segmentation.tex
% 
%     Section tag and label are placed in this top file.
%
%
%
%%%%%%%%%%%%%%%%%%%%%%%%%%%%%%%%%%%%%%%%%%%%%%%%%%%%%%%%%%%%%%%%%%%%%%%%



\itkpiccaption[Zero Set Concept]{Concept of zero set in a level set.\label{fig:LevelSetZeroSet}}
\parpic(9cm,6cm)[r]{\includegraphics[width=8cm]{LevelSetZeroSet.eps}}

The paradigm of the level set is that it is a numerical method for tracking
the evolution of contours and surfaces. Instead of manipulating the contour
directly, the contour is embedded as the zero level set of a higher
dimensional function called the level-set function, $\psi(\bf{X},t)$. The
level-set function is then evolved under the control of a differential
equation.  At any time, the evolving contour can be obtained by extracting
the zero level-set $\Gamma(\bf(X),t) =
\{\psi(\bf{X},t) = 0\}$ from the output.  The main advantages of using level
sets is that arbitrarily complex shapes can be modeled and topological
changes such as merging and splitting are handled implicitly. 

Level sets can be used for image segmentation by using image-based features
such as mean intensity, gradient and edges in the governing differential
equation.  In a typical approach, a contour is initialized by a user and is
then evolved until it fits the form of an object in the image.
Many different implementations and variants of this basic concept have been
published in the literature. An overview of the field has been made by
Sethian \cite{Sethian1996}.

The following sections introduce practical examples of some
of the level set segmentation methods available in ITK.  The remainder of this
section describes features common to all of these filters except the
\doxygen{itk}{FastMarchingImageFilter}, which is derived from a different code
framework.  Understanding these features will aid in using the filters
more effectively.

Each filter makes use of a generic level-set equation to compute the update to
the solution $\psi$ of the partial differential equation.

\begin{equation}
\label{eqn:LevelSetEquation}
\frac{d}{dt}\psi = -\alpha \mathbf{A}(\mathbf{x})\cdot\nabla\psi - \beta
  P(\mathbf{x})\mid\nabla\psi\mid + 
\gamma Z(\mathbf{x})\kappa\mid\nabla\psi\mid
\end{equation}
 
where $\mathbf{A}$ is an advection term, $P$ is a propagation (expansion) term,
and $Z$ is a spatial modifier term for the mean curvature $\kappa$.  The scalar
constants $\alpha$, $\beta$, and $\gamma$ weight the relative influence of
each of the terms on the movement of the interface.  A segmentation filter may
use all of these terms in its calculations, or it may omit one or more terms.
If a term is left out of the equation, then setting the corresponding scalar
constant weighting will have no effect.

All of the level-set based segmentation filters \emph{must} operate with
floating point precision to produce valid results.  The third, optional
template parameter is the \emph{numerical type} used for calculations and as
the output image pixel type.  The numerical type is \code{float} by default,
but can be changed to \code{double} for extra precision.  A user-defined,
signed floating point type that defines all of the necessary arithmetic
operators and has sufficient precision is also a valid choice.  You should
not use types such as \code{int} or \code{unsigned char} for the numerical
parameter.  If the input image pixel types do not match the numerical type,
those inputs will be cast to an image of appropriate type when the filter is
executed.

Most filters require two images as input, an initial model $\psi(\bf{X},
t=0)$, and a \emph{feature image}, which is either the image you wish to
segment or some preprocessed version.  You must specify the isovalue that
represents the surface $\Gamma$ in your initial model. The single image
output of each filter is the function $\psi$ at the final time step.  It is
important to note that the contour representing the surface $\Gamma$ is the
zero level-set of the output image, and not the isovalue you specified for
the initial model.  To represent $\Gamma$ using the original isovalue, simply
add that value back to the output.

The solution $\Gamma$ is calculated to subpixel precision.  The best discrete
approximation of the surface is therefore the set of grid positions closest to
the zero-crossings in the image, as shown in
Figure~\ref{fig:LevelSetSegmentationFigure1}.  The
\doxygen{itk}{ZeroCrossingImageFilter} operates by finding exactly those grid 
positions and can be used to extract the surface. 


\begin{figure}
\centering
\includegraphics[width=0.4\textwidth]{LevelSetSegmentationFigure1.eps}
\itkcaption[Grid position of the embedded level-set surface.]{The implicit level
set surface $\Gamma$ is the black line superimposed over the image grid.  The location
of the surface is interpolated by the image pixel values.  The grid pixels
closest to the implicit surface are shown in gray. }
\protect\label{fig:LevelSetSegmentationFigure1}
\end{figure}

There are two important considerations when analyzing the processing time for
any particular level-set segmentation task: the surface area of the evolving
interface and the total distance that the surface must travel.  Because the
level-set equations are usually solved only at pixels near the surface (fast
marching methods are an exception), the time taken at each iteration depends on
the number of points on the surface.  This means that as the surface grows, the
solver will slow down proportionally.  Because the surface must evolve slowly
to prevent numerical instabilities in the solution, the distance the surface
must travel in the image dictates the total number of iterations required.

Some level-set techniques are relatively insensitive to initial conditions
and are therefore suitable for region-growing segmentation. Other techniques,
such as the \doxygen{itk}{LaplacianSegmentationLevelSetImageFilter}, can easily
become ``stuck'' on image features close to their initialization and should
be used only when a reasonable prior segmentation is available as the
initialization.  For best efficiency, your initial model of the surface
should be the best guess possible for the solution. 


\subsection{Fast Marching Segmentation}
\label{sec:FastMarchingImageFilter}

\ifitkFullVersion
\input{FastMarchingImageFilter.tex}
\fi


%% \subsection{Shape Detection Segmentation}
%% \label{sec:ShapeDetectionLevelSetFilter}

%% \ifitkFullVersion
%% \input{ShapeDetectionLevelSetFilter.tex}
%% \fi


%% \subsection{Geodesic Active Contours Segmentation}
%% \label{sec:GeodesicActiveContourImageFilter}

%% \ifitkFullVersion
%% \input{GeodesicActiveContourImageFilter.tex}
%% \fi


%% \subsection{Threshold Level Set Segmentation}
%% \label{sec:ThresholdSegmentationLevelSetImageFilter}
%% \ifitkFullVersion
%% \input{ThresholdSegmentationLevelSetImageFilter.tex}
%% \fi


%% \subsection{Canny-Edge Level Set Segmentation}
%% \label{sec:CannySegmentationLevelSetImageFilter}
%% \ifitkFullVersion
%% \input{CannySegmentationLevelSetImageFilter.tex}
%% \fi


%% \subsection{Laplacian Level Set Segmentation}
%% \label{sec:LaplacianSegmentationLevelSetImageFilter}
%% \ifitkFullVersion
%% \input{LaplacianSegmentationLevelSetImageFilter.tex}
%% \fi

%% \subsection{Geodesic Active Contours Segmentation With Shape Guidance}
%% \label{sec:GeodesicActiveContourShapePriorLevelSetImageFilter}
%% \ifitkFullVersion
%% \input{GeodesicActiveContourShapePriorLevelSetImageFilter.tex}
%% \fi



\fi


%% \section{Hybrid Methods} 
%% \label{sec:HybridSegmentationMethods}

%% \ifitkFullVersion 
%% %%%%%%%%%%%%%%%%%%%%%%%%%%%%%%%%%%%%%%%%%%%%%%%%%%%%%%%%%%%%%%%%%%%%%%%%
%
%
%     This file is included from the file   Segmentation.tex
% 
%     Section tag and label are placed in this top file.
%
%
%
%%%%%%%%%%%%%%%%%%%%%%%%%%%%%%%%%%%%%%%%%%%%%%%%%%%%%%%%%%%%%%%%%%%%%%%%

\subsection{Introduction}
\label{sec:HybridSegmentationIntroduction}
This section introduces the use of hybrid methods for segmentation of image
data. The hybrid segmentation approach integrates boundary-based and
region-based segmentation methods that amplify the strength but reduce the
weakness of both techniques. The advantage of this approach comes from
combining region-based segmentation methods like the fuzzy connectedness and
Voronoi diagram classification with boundary-based deformable model
segmentation. The synergy between fundamentally different methodologies tends
to result in robustness and higher segmentation quality.  A hybrid
segmentation engine can be built, as illustrated in
Figure~\ref{fig:ComponentsofaHybridSegmentationApproach}. It consists of
modules representing segmentation methods and implemented as ITK filters. We
can derive a variety of hybrid segmentation methods by exchanging the filter
used in each module. It should be noted that under the fuzzy connectedness
and deformable models modules, there are several different filters that can
be used as components. Below, we describe two examples of hybrid segmentation
methods, derived from the hybrid segmentation engine: integration of fuzzy
connectedness and Voronoi diagram classification (hybrid method 1), and
integration of Gibbs prior and deformable models (hybrid method 2).  Details
regarding the concepts behind these methods have been discussed in the
literature
\cite{Angelini2002,Udupa2002,Jin2002,Imielinska2001,Imielinska2000a,Imielinska2000b}


\subsection{Fuzzy Connectedness and Confidence Connectedness }

Probably the simplest combination of hybrid filters is the pair formed by the
\doxygen{ConfidenceConnectednessImageFilter} and
\doxygen{SimpleFuzzyConnectednessScalarImageFilter}. In this combination the
confidence connectedness filter is used to produce a rough segmentation of an
object and to compute and estimation of the mean and variance of
gray values in such structure. The values of mean and variance are then passed
to the Simple Fuzzy Connectedness image filter in order to compute an affinity map.

\ifitkFullVersion
\input{FuzzyConnectednessImageFilter.tex}
\fi



\subsection{Fuzzy Connectedness and Voronoi Classification}
\label{sec:HybridMethod1}
In this section we present a hybrid segmentation method that requires
minimal manual initialization by integrating the fuzzy connectedness and
Voronoi diagram classification segmentation algorithms. We start with a fuzzy
connectedness filter to generate a sample of tissue from a region to be
segmented. From the sample, we automatically derive image statistics that
constitute the homogeneity operator to be used in the next stage of the
method. The output of the fuzzy connectedness filter is used as a prior to
the Voronoi diagram classification filter. This filter performs iterative
subdivision and classification of the segmented image resulting in an
estimation of the boundary. The output of this filter is a 3D binary image
that can be used to display the 3D result of the segmentation, or passed to
another filter (e.g. deformable model) for further improvement of the final
segmentation. Details describing the concepts behind these methods have been
published in
\cite{Angelini2002,Udupa2002,Jin2002,Imielinska2001,Imielinska2000a,Imielinska2000b}

In Figure~\ref{fig:UMLClassDiagramoftherFuzzyConnectednessFilter}, we
describe the base class for simple fuzzy connectedness segmentation. This
method is non-scale based and non-iterative, and requires only one seed to
initialize it. We define affinity between two nearby elements in a image
(e.g. pixels, voxels) via a degree of adjacency, similarity in their
intensity values, and their similarity to the estimated object.  The closer
the elements and the more similar their intensities, the greater the affinity
between them. We compute the strength of a path and fuzzy connectedness
between each two pixels (voxels) in the segmented image from the fuzzy
affinity.  Computation of the fuzzy connectedness value of each pixel (voxel)
is implemented by selecting a seed point and using dynamic programming. The
result constitutes the fuzzy map. Thresholding of the fuzzy map gives a
segmented object that is strongly connected to the seed point (for more
details, see \cite{Udupa1996}). Two fuzzy connectedness filters are available
in the toolkit:

\begin{itemize}
\item The \doxygen{SimpleFuzzyConnectednessScalarImageFilter}, an implementation of the
fuzzy connectedness segmentation of single-channel (grayscale) image.
\item The \doxygen{SimpleFuzzyConnectednessRGBImageFilter}, an implementation
of fuzzy connectedness segmentation of a three-channel (RGB) image.
\end{itemize}

New classes can be derived from the base class by defining other affinity
functions and targeting multi-channel images with an arbitrary number of
channels. Note that the simple fuzzy connectedness filter can be used as a
stand-alone segmentation method and does not necessarily need to be combined
with other methods as indicated by
Figure~\ref{fig:UMLCollaborationDiagramoftheFuzzyConnectednessFilter}.

In Figure~\ref{fig:UMLVoronoiSegmentationClassFilter} we present the base
class for Voronoi diagram classification. We initialize the method with a
number of random seed points and compute the Voronoi diagram over the
segmented 2D image. Each Voronoi region in the subdivision is classified as
internal or external, based on the homogeneity operator derived from the
fuzzy connectedness algorithm.  We define boundary regions as the external
regions that are adjacent to the internal regions.  We further subdivide the
boundary regions by adding seed points to them. We converge to the final
segmentation using simple stopping criteria (for details, see
\cite{Imielinska2001}). Two Voronoi-based segmentation methods are available in ITK: the \doxygen{VoronoiSegmentationImageFilter} for processing single-channel
(grayscale) images, and the \doxygen{VoronoiSegmentationRGBImageFilter}, for
segmenting three-channel (RGB) images. New classes can be derived from the
base class by defining other homogeneity measurements and targeting
multichannel images with an arbitrary number of channels.  The other classes
that are used for computing a 2D Voronoi diagram are shown in
Figure~\ref{fig:UMLClassesforImplementationofVoronoiDiagramFilter}. Note that
the Voronoi diagram filter can be used as a stand-alone segmentation method,
as depicted in
Figure~\ref{fig:UMLCollaborationDiagramoftheVoronoiSegmentationFilter}.

Figures~\ref{fig:UMLHybridMethodDiagram1} and 
\ref{fig:UMLHybridMethodDiagram2} illustrate hybrid segmentation methods
that integrate fuzzy connectedness with Voronoi diagrams, and fuzzy
connectedness, Voronoi diagrams and deformable models, respectively.


\begin{figure}
\center
\includegraphics[width=0.8\textwidth]{HybridSegmentationEngine1.eps}
\itkcaption[Hybrid Segmentation Engine]{The hybrid segmentation engine.}
\label{fig:ComponentsofaHybridSegmentationApproach}
\end{figure}


\begin{figure}
\center
\includegraphics[width=0.8\textwidth]{FuzzyConnectednessClassDiagram1.eps}
\itkcaption[FuzzyConectedness Filter Diagram]{Inheritance diagram for the fuzzy connectedness filter.}
\label{fig:UMLClassDiagramoftherFuzzyConnectednessFilter}
\end{figure}


\begin{figure}
\center
\includegraphics[width=0.8\textwidth]{FuzzyConnectednessCollaborationDiagram1.eps}
\itkcaption[Fuzzy Connectedness Segmentation Diagram]{Inputs and outputs to
FuzzyConnectednessImageFilter segmentation algorithm.}
\label{fig:UMLCollaborationDiagramoftheFuzzyConnectednessFilter}
\end{figure}

\begin{figure}
\center
\includegraphics[width=0.8\textwidth]{VoronoiSegmentationClassDiagram1.eps}
\itkcaption[Voronoi Filter class diagram]{Inheritance diagram for the Voronoi
segmentation filters.}
\label{fig:UMLVoronoiSegmentationClassFilter}
\end{figure}

\begin{figure}
\center
\includegraphics[width=0.8\textwidth]{VoronoiSegmentationCollaborationDiagram1.eps}
\itkcaption[Voronoi Diagram Filter classes]{Classes used by the Voronoi 
segmentation filters.}
\label{fig:UMLClassesforImplementationofVoronoiDiagramFilter}
\end{figure}


\begin{figure}
\center
\includegraphics[width=0.8\textwidth]{VoronoiSegmentationCollaborationDiagram2.eps}
\itkcaption[Voronoi Diagram Segmentation]{Input and output to the 
VoronoiSegmentationImageFilter.}
\label{fig:UMLCollaborationDiagramoftheVoronoiSegmentationFilter}
\end{figure}


\begin{figure}
\center
\includegraphics[width=0.8\textwidth]{FuzzyVoronoiCollaborationDiagram1.eps}
\itkcaption[Fuzzy Connectedness and Voronoi Diagram Classification]{Integration
 of the fuzzy connectedness and Voronoi segmentation filters.}
\label{fig:UMLHybridMethodDiagram1}
\end{figure}

\begin{figure}
\center
\includegraphics[width=0.8\textwidth]{FuzzyVoronoiDeformableCollaborationDiagram1.eps}
\itkcaption[Fuzzy Connectedness, Voronoi diagram, and Deformable
Models]{Integration of the fuzzy connectedness, Voronoi, and 
deformable model segmentation methods.}
\label{fig:UMLHybridMethodDiagram2}
\end{figure}



\subsubsection{Example of a Hybrid Segmentation Method}
\label{sec:HybridMethod1:Example}

%\ifitkFullVersion
%\input{HybridSegmentationFuzzyVoronoi.tex}
%\fi




%% \subsection{Deformable Models and Gibbs Prior}

%% Another combination that can be used in a hybrid segmentation method is the
%% set of Gibbs prior filters with deformable models.

%% \subsubsection{Deformable Model}
%% \ifitkFullVersion
%% \input{DeformableModel1.tex}
%% \fi


%% \subsubsection{Gibbs Prior Image Filter}
%% \ifitkFullVersion
%% \input{GibbsPriorImageFilter1.tex}
%% \fi


%% \fi




