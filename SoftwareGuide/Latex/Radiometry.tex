\chapter{Radiometry}

Remote sensing is not just a matter of taking pictures, but also --
mostly -- a matter of measuring physical values. In order to properly
deal with physical magnitudes, the numerical values provided by the
sensors have to be calibrated. After that, several indices with
physical meaning can be computed.

Calibration functionnalities (absolute and relative) and even
atmospheric correction routines will be available in future versions
of OTB. Please note that the 6S Radiative Transfer Code\footnote{\url{http://6s.ltdri.org/}} is already included in the OTB source code and
compiles out of the box. Calibration and atmospheric corrections in
OTB will be based on it.

In the current version of OTB, several vegetation indices are already
available. They are presented in this chapter.


%%%%%%%%%%%%%%%%%%%%%%%%%%%%%%%%%%%%%%%%%%%%%%%%%%%%%%%%%%%%%%%%%%%%%%
\section{Vegetation Index}
\label{sec:VegetationIndex}

\subsection{Introduction}
A vegetation index is a quantitative measure used to measure biomass
or vegetative vigor, usually formed from combinations of several
spectral bands, whose values are added, divided, or multiplied in
order to yield a single value that indicates the amount or vigor of
vegetation.

\subsection{NDVI}
\label{secNDVI}
NDVI was one of the most successful of many attempts to simply and
quickly identify vegetated areas and their {\em condition}, and it remains
the most well-known and used index to detect live green plant canopies
in multispectral remote sensing data. Once the feasibility to detect
vegetation had been demonstrated, users tended to also use the NDVI to
quantify the photosynthetic capacity of plant canopies. This, however,
can be a rather more complex undertaking if not done properly.
\input{NDVIRAndNIRVegetationIndexImageFilter.tex}

\subsection{ARVI}
\label{secARVI}
\input{ARVIMultiChannelRAndBAndNIRVegetationIndexImageFilter.tex}



\section{Atmospheric Corrections}
\label{secAtmosphericCorrections}
\input{AtmosphericCorrectionSequencement.tex}
