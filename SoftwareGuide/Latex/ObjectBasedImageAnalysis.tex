
\chapter{Object-based Image Analysis }\label{sec:OBIA}

Object-Based Image Analysis (OBIA) consists in analysing images at
the object level instead of working at the pixel level. This approach
is particularly well adapted for high resolution images and allows
more robust and less noisy results.

OTB allows to implement OBIA by using ITK's Label Object framework
(\url{http://www.insight-journal.org/browse/publication/176}). This
allows to represent a segmented image as a set of regions and not
anymore as a set of pixels. Added to the compression rate achieved by
this kind of description, the main advantage of this approach is the
possibility to operate at the segment (or object level).

A classical OBIA pipeline will use the following steps:

\begin{enumerate}
\item Image segmentation (the whole or only parts of it).
\item Image to LabelObjectMap (a kind of \code{std::map<LabelObject>}) transformation.
\item Eventual relabeling.
\item Attribute computation for the regions using the image before
  segmentation:
  \begin{enumerate}
         \item Shape attributes.
         \item Statistics attributes.
         \item Attributes for radiometry, textures, etc.
  \end{enumerate}

\item Object filtering
  \begin{enumerate}
         \item Remove/select objects under a condition (area less than
         X, NDVI higher than X, etc.)
	 \item Keep N objects.
	 \item etc.
  \end{enumerate}
\item LabelObjectMap to image transformation.
\end{enumerate}


\section{From Images to Objects}\label{sec:FromImagesToObjects}
\input{ImageToLabelToImage.tex}

\section{Object Attributes}\label{sec:ObjectAttributes}
\input{ShapeAttributeComputation.tex}


\section{Object Filtering}

\section{Object-based classification}
