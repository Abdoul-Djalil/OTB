\chapter{Multi-scale Analysis}
\section{Introduction}

In this chapter, the tools for multi-scale and multi-resoltuion
processing (analysis, synthesis and fusion) will be presented. Most of
the algorithms are based on pyramidal approaches. These approaches
were first used for image compression and they are based on the fact
that, once an image has been low-pass filtered it does not have
details beyond the cut-off frequency of the low-pass
filter any more. Therefore, the image can be subsampled -- decimated -- without
any loss of information.

A pyramidal decomposition is thus performed applying the following 3
steps in an iterative way:
\begin{enumerate}
  \item Low pas filter the image $I_{n}$ in order to produce $F(I_n)$;
  \item Compute the difference $D_n = I_n - F(I_n)$ which corresponds
  to the details at level $n$;
  \item Subsample $F_(I_n)$ in order to obtain $I_{n+1}$.
\end{enumerate}

The result is a series of decrasing resolution images $I_k$ and a
series of decreasing resolution details $D_k$.



\section{Morphological Pyramid}\label{secMorphoPyr}
If the smoothing filter used in the pyramidal analysis is a
morphological filter, one cannot safely subsample the filtered image
without loss of information. However, by keeping the details possibly
lost in the down-sampling operation, such a decomposition can be used.

The Morphological Pyramid is an approach to such a
decomposition. It's computation process is an iterative analysis
involving smoothing by the morphological filter, computing the
details lost in the smoothing, down-sampling the current image, and
computing the details lost in the down-sampling.

\input{MorphologicalPyramidAnalyseFilterExample.tex}

\input{MorphologicalPyramidSynthesisFilterExample.tex}


\subsection{Morphological Pyramid Exploitation}
One of the possible uses of the morphological pyramid is the
segmentation of objects -- regions -- of a particular scale.

\input{MorphologicalPyramidSegmenterExample.tex}

This same approach can be applied to all the levels of the
morphological pyramid analysis.

\input{MorphologicalPyramidSegmentationExample.tex}