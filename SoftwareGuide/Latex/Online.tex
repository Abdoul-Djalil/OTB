\chapter{Online data}\label{sec:Online}

With almost every computer connected to the internet, the amount
of information online is growing. It is quite easy to retrieve some
valuable information. OTB has a few experimental classes for this purpose.

For these examples to work, you need to have OTB compiled with the
\texttt{OTB\_USE\_CURL} option to \texttt{ON}.

Let's see what we can do.

\section{Name to coordinates}
\label{sec:NamesToCoordinates}
\input{PlaceNameToLonLatExample.tex}


\section{Open street map}
\label{sec:OpenStreetMap}

The power of sharing which is a driving force in open source programs such
as OTB can also be demonstrated for data collection. One good example is
Open Street Map (\url{www.openstreetmap.org/}).

In this project, thousands of users, upload GPS data and draw map of their
surroundings. The coverage is impressive and this data is made freely available.

It is possible of course to get the vector data (not covered yet by otb), but
here we will focuss on retrieving some nice map for any place. The following
example describe the method. This part is pretty experimental and the code is
not as polished as the rest of the library. You've been warned!

\input{TileMapImageIOExample.tex}

