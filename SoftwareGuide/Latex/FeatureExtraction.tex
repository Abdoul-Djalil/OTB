\chapter{Feature Extraction}

% \section{Introduction}

Under the term {\em Feature Extraction} we include several techniques
aiming to detect or extract informations of low level of abstraction
from images. These {\em features} can be objects : points, lines,
etc. They can also be measures : moments, textures, etc.

\section{Textures}
\subsection{Haralick Descriptors}
\input{TextureExample}

\subsection{PanTex}
\input{PanTexExample}

\section{Interest Points}
\subsection{Harris detector}
\input{HarrisExample}
\subsection{SIFT detector}
\label{sec:SIFTDetector}
\input{SIFTFastExample}
\subsection{SURF detector}
\input{SURFExample}

\section{Alignments}
\label{sec:Alignments}
\input{AlignmentsExample}
\section{Lines}
\label{sec:LineDetectors}

\subsection{Line Detection}
\label{sec:LineDetection}
\input{RatioLineDetectorExample}
\input{CorrelationLineDetectorExample}
\input{AssymmetricFusionOfLineDetectorExample}
\input{ParallelLineDetectionExample}


\subsection{Segment Extraction}
\label{sec:SegmentExtraction}
\subsubsection{Local Hough Transform}
\input{LocalHoughExample}
%\input{ExtractSegmentsByStepsExample}
%\input{ExtractSegmentsExample}
\subsubsection{Line Segment Detector}
\label{sec:LSD}
\input{LineSegmentDetectorExample}
\subsection{Right Angle Detector}
\label{sec:RightAngleDetector}
\input{RightAngleDetectionExample}


\section{Density Features}
An interesting approach to feature extraction consists in computing
the density of previously detected features as simple edges or
interest points.
\subsection{Edge Density}
\input{EdgeDensityExample}
\subsection{SIFT Density}
\input{SIFTDensityExample}

\section{Geometric Moments}

\subsection{Complex Moments}
\label{sec:ComplexMoments}
The complex geometric moments are defined as:
\begin {equation}
c_{pq} = \int\limits_{-\infty}^{+\infty}\int\limits_{-\infty}^{+\infty}(x + iy)^p(x- iy)^qf(x,y)dxdy,
\label{2.2}
\end{equation}
where $x$ and $y$ are the coordinates of the image $f(x,y)$, $i$ is the
imaginary unit and
$p+q$ is the order of $c_{pq}$. The geometric moments are
particularly useful in the case of scale changes.

\subsubsection{Complex Moments for Images}
\input{ComplexMomentImageExample}
\subsubsection{Complex Moments for Paths}
\input{ComplexMomentPathExample}

\subsection{Hu Moments}
\label{sec:HuMoments}
Using the algebraic moment theory, H. Ming-Kuel obtained a family of 7
invariants with respect to planar transformations called Hu invariants,
\cite{hu}. Those invariants can be seen as nonlinear combinations of
the complex moments. Hu invariants have
been very much used in object recognition during the last 30 years,
since they are invariant to rotation, scaling and translation. \cite{flusserinv} gives their expressions :

\begin{equation}
\begin{array}{cccc}
\phi_1 = c_{11};& \phi_2 = c_{20}c_{02};& \phi_3 = c_{30}c_{03};& \phi_4 = c_{21}c_{12};\\
\phi_5 = Re(c_{30}c_{12}^3);& \phi_6 = Re(c_{21}c_{12}^2);& \phi_7 = Im(c_{30}c_{12}^3).&\\
\end{array}
\end{equation}


\cite{dudani} have used these invariants for the recognition of
aircraft silhouettes. Flusser and Suk have used them for image
registration, \cite{flusser_2}.

\subsubsection{Hu Moments for Images}
\input{HuMomentImageExample}
%\subsubsection{Hu Moments for Paths}
%\input{HuMomentPathExample}


\subsection{Flusser Moments}
\label{sec:FlusserMoments}
The Hu invariants have been modified and
improved by several authors. Flusser used these moments in order to
produce a new family of descriptors of order higher than 3,
\cite{flusserinv}. These descriptors are invariant to scale and
rotation. They have the following expressions:
\begin {equation}
\begin{array}{ccc}
\psi_1  = c_{11} = \phi_1; &  \psi_2  = c_{21}c_{12} = \phi_4; & \psi_3  = Re(c_{20}c_{12}^2) = \phi_6;\\
\psi_4  = Im(c_{20}c_{12}^2); & \psi_5  = Re(c_{30}c_{12}^3) = \phi_5;
& \psi_6  = Im(c_{30}c_{12}^3) = \phi_7.\\
\psi_7  = c_{22}; & \psi_8  = Re(c_{31}c_{12}^2); & \psi_9  = Im(c_{31}c_{12}~2);\\
\psi_{10} = Re(c_{40}c_{12}^4); & \psi_{11} = Im(c_{40}c_{12}^2). &\\

\end{array}
\end {equation}

\textbf{Examples}
\subsubsection{Flusser Moments for Images}
\input{FlusserMomentImageExample}
%\subsubsection{Flusser Moments for Paths}
%\input{FlusserMomentPathExample}

\section{Principal Component Analysis}
Principal Component Analysis is a linear projection technique which
can be used for dimensionality reduction.
\input{InnerProductPCAExample}

\section{Road extraction}
\label{sec:RoadExtraction}

Road extraction is a critical feature for an efficient use of high resolution satellite images. There are many applications of road extraction: update of GIS database, reference for image registration, help for identification algorithms and rapid mapping for example.  Road network can be used to register an optical image with a map or an optical image with a radar image for example. Road network extraction can help for other algorithms: isolated building detection, bridge detection. In these cases, a rough extraction can be sufficient. In the context of response to crisis, a fast mapping is necessary: within 6~hours, infrastructures for the designated area are required. Within this timeframe, a manual extraction is inconceivable and an automatic help is necessary.

\subsection{Road extraction filter}

\input{ExtractRoadExample}

\subsection{Step by step road extraction}

\input{ExtractRoadByStepsExample}


% \section{Seam carving}
% \label{sec:SeamCarving}
%
% \input{SeamCarvingExample}
%
% \input{SeamCarvingOtherExample}

\section{Cloud Detection}
\input{CloudDetectionExample}