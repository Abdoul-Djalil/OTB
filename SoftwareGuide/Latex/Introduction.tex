\chapter{Welcome}
\label{chapter:Welcome}

Welcome to the \emph{ORFEO ToolBox (OTB) Software Guide}.

This document presents the essential concepts used in OTB. It will
guide you through the road of learning and using OTB. The Doxygen
documentation for the OTB application programming interface is
available on line at \url{http://orfeo-toolbox.sourceforge.net/Doxygen/html}.

\section{Organization}
\label{sec:Organization}

This software guide is divided into three parts, each of which is further
divided into several chapters. Part I is a general introduction to OTB,
with---in the next chapter---a description of how to install the ORFEO
Toolbox on your computer. Part I also
introduces basic system concepts such as an overview of the system
architecture, and how to build applications in the C++ programming
language. Part II describes the system from the user point of view. Dozens
of examples are used to illustrate important system features. Part III is for
the OTB developer. It explains how to create your own classes and extend
the system.%, and interface to windowing and GUI systems.

\section{How to Learn OTB}
\label{sec:HowToLearnOTB}

There are two broad categories of users of OTB. First are class
developers, those who create classes in C++. The second, users, employ
existing C++ classes to build applications. Class developers must be
proficient in C++, and if they are extending or modifying OTB, they
must also be familiar with OTB's internal structures and design
(material covered in Part III). 

The key to learning how to use OTB is to become familiar with its
palette of objects and the ways of combining them. We recommend that you learn the system by studying the
examples and then, if you are a class developer, study the source
code. Start by reading Chapter \ref{chapter:SystemOverview}, which provides an overview of some
of the key concepts in the system, and then review the examples in
Part II. You may also wish to compile and run the dozens of examples
distributed with the source code found in the directory
\code{OTB/Examples}. (Please see the file
\code{OTB/Examples/README.txt} for a description of the examples
contained in the various subdirectories.) There are also several
hundreds of tests found in the source distribution in
\code{OTB/Testing/Code}, most of which are minimally documented
testing code. However, they may be useful to see how classes are used
together in OTB, especially since they are designed to exercise as
much of the functionality of each class as possible.

\section{Software Organization}
\label{sec:SoftwareOrganization}

The following sections describe the directory contents, summarize the
software functionality in each directory, and locate the documentation and
data.

\subsection{Obtaining the Software}
\label{sec:ObtainingTheSoftware}

Periodic releases of the software are available on the OTB Web
site. These official releases are available a few times a year and
announced on the ORFEO Web pages and mailing lists. 

This software guide assumes that you are working with the official OTB
version 1.0 release (available on the OTB Web site). %If you are a new user,
%we highly recommend that you use the released version of the software.

\section{Downloading OTB}
\label{sec:DownloadingOTB}
 
\index{Downloading}

OTB can be downloaded without cost from the following web site:
\begin{center} 
  \url{http://otb.cnes.fr/}
\end{center}
In order to track the kind of applications for which OTB is being used, you
will be asked to complete a form prior to downloading the software.
The information you provide in this form will help developers to get a better
idea of the interests and skills of the toolkit users.

Once you fill out this form you will have access to the download
page. This page can be book marked to facilitate subsequent visits to
the download site without having to complete any form again.

 Then choose the tarball that better fits your system. The options
are \code{.zip} and \code{.tgz} files.  The first type is better suited for
MS-Windows while the second one is the preferred format for UNIX systems.

Once you unzip or untar the file, a directory called \code{OTB} will be
created in your disk and you will be ready for starting the configuration
process described in Section \ref{sec:CMakeforOTB} on page 
\pageref{sec:CMakeforOTB}.


\subsection{Join the Mailing List}
\label{sec:JoinMailList}

\index{OTB!mailing list}
\index{mailing list}

It is strongly recommended that you join the users mailing list. This is one
of the primary resources for guidance and help regarding the use of the 
toolkit. You can subscribe to the users list online at

\begin{center}
\url{http://groups.google.com/group/otb-users}
\end{center} 

The otb-users mailing list is also the best mechanism for expressing your
opinions about the toolbox and to let developers know about features that you
find useful, desirable or even unnecessary. OTB developers are committed to
creating a self-sustaining open-source OTB community. Feedback from users is
fundamental to achieving this goal.

\subsection{Directory Structure}
\label{sec:DirectoryStructure}

To begin your OTB odyssey, you will first need to know something about OTB's
software organization and directory structure. It is helpful to know enough to navigate through the
code base to find examples, code, and documentation.

OTB is organized into several different modules. There are three:
the \code{OTB}, \code{OTB-Documents} and \code{OTB-Applications}
modules. The source code, examples and applications are found in the
\code{OTB} module; documentation, tutorials, and material related to the
design and marketing of OTB are found in \code{OTB-Documents}; and fairly
complex applications using OTB (and other systems such as VTK and FLTK)
are available from \code{OTB-Applications}. Usually you will work with the
\code{OTB} module unless you are a developer, are teaching a course, or
are looking at the details of various design documents. The
\code{OTB-Applications} module should only be downloaded and
compiled once
the \code{OTB} module is functioning properly.

The \code{OTB} module contains the following subdirectories:
\begin{itemize}
        \item \code{OTB/Code}---the heart of the software; the location
        of the majority of the source code.
        \item \code{OTB/Examples}---a suite of simple, well-documented
        examples used by this guide and to illustrate important
        OTB concepts.
        \item \code{OTB/Testing}---a large number of small programs used
        to test OTB. These examples tend to be minimally documented but may
        be useful to demonstrate various system concepts. 
        \item \code{OTB/Utilities}---supporting software for the OTB
        source code. For example, libraries such as \code{ITK}.% and \code{GDAL}.
\end{itemize}

The source code directory structure---found in \code{OTB/Code}---is 
important to understand since other directory structures (such as the
\code{Testing} directory) shadow the structure of the
\code{OTB/Code} directory.
\begin{itemize}
        \item \code{OTB/Code/Common}---core classes, macro definitions,
        typedefs, and other software constructs central to OTB.
        \item \code{OTB/Code/BasicFilters}---basic image processing
        filters.
        \item \code{OTB/Code/IO}---classes that support the reading
        and writing of data.
	\item \code{OTB/Code/FeatureExtraction}---the location of many
	feature extraction algorithms.
	\item \code{OTB/Code/ChangeDetection}---a set of remote
	sensing image change detection algorithms.
	\item \code{OTB/Code/MultiScale}---a set of functionalities
	for multiscale image analysis and synthesis.
	\item \code{OTB/Code/Learning}---several functionnalities for
	supervised learning and classification.
	\item \code{OTB/Code/SpatialReasoning}---several
	functionnalities high level image analysis using spatial
	reasoning techniques.
	\item \code{OTB/Code/Visu}---utilities for simple image
	visualization.
	\item \code{OTB/Code/Gui}---very basic widgets for building
	graphical user interfaces, such as progress bars for filters, etc.
	  
\end{itemize}

The \code{OTB-Documents} module contains the following subdirectories:
\begin{itemize}
        \item \code{OTB-Documents/CourseWare}---material related to
        teaching OTB.
        \item \code{OTB-Documents/Latex}---\LaTeX{} styles to produce this
        work as well as other documents.
        %\item \code{OTB-Documents/Marketing}---marketing flyers and 
        %literature used to succinctly describe OTB.
        %\item \code{OTB-Documents/Papers}---papers related to the many
        %algorithms, data representations, and software tools used in OTB.
        \item \code{OTB-Documents/SoftwareGuide}---\LaTeX{} files used to
        create this guide. (Note that the code found in
        \code{OTB/Examples} is used in conjunction with these \LaTeX{}
        files.)
        %\item \code{OTB-Documents/Validation}---validation case studies
        %using OTB.
\end{itemize}

The \code{OTB-Applications} module contains large, relatively complex
examples of OTB usage.%%  See the web pages at 
%% \url{http://www.itk.org/HTML/Applications.htm}\textbf{FIXME} for a description. Some of 
%% these applications require GUI toolkits such as FLTK or other packages
%% such as VTK (\emph{The Visualization Toolkit}
%% \url{http://www.vtk.org}). Do not attempt to compile and build this module
%% until you have successfully built the core \code{OTB} module.


\subsection{Documentation}
\label{sec:Documentation}

Besides this text, there are other documentation resources that you should be
aware of.
\begin{description}
        \item[Doxygen Documentation.] The Doxygen documentation is an
        essential resource when working with OTB. These extensive Web
        pages describe in detail every class and method in the
        system. The documentation also contains inheritance and
        collaboration diagrams, listing of event invocations, and data
        members. The documentation is heavily hyper-linked to other
        classes and to the source code. The Doxygen documentation is
        available on-line at
        \url{http://orfeo-toolbox.sourceforge.net/Doxygen/html}.

	\item[Header Files.] Each OTB class is implemented with a .h and
        .cxx/.txx file (.txx file for templated classes). All methods
        found in the .h header files are documented and provide a quick way
        to find documentation for a particular method. (Indeed, Doxygen uses
        the header documentation to produces its output.)
\end{description}

\subsection{Data}
\label{sec:Data}

The OTB Toolkit was designed to support the ORFEO Acompaniment Program 
and its associated data. This data is available \url{http://smsc.cnes.fr/PLEIADES/index.htm}.


\section{The OTB Community and Support}
\label{sec:AdditionalResources}

OTB was created from its inception as a collaborative, community
effort. Research, teaching, and commercial uses of the toolkit are
expected. If you would like to participate in the community, there are a
number of possibilities.

\begin{itemize}
       \item Users may actively report bugs, defects in the system API,
       and/or submit feature requests. Currently the best way to do this is
       through the OTB users mailing list.

       \item Developers may contribute classes or improve existing
       classes. If you are a developer, you may request permission to join
       the OTB developers mailing list. Please do so by sending email to
       otb ``at'' cnes.fr. To become a developer you need to
       demonstrate both a level of competence as well as
       trustworthiness. You may wish to begin by submitting fixes to the OTB
       users mailing list.

       \item Research partnerships with members of the ORFEO
       Acompaniment Program are encouraged. CNES will encourage the use of
       OTB in proposed work and research projects.

%%        \item For those developing commercial applications with OTB,
%%        support and consulting are available from Kitware at
%%        \url{http://www.kitware.com}. Kitware also offers short OTB courses
%%        either at a site of your choice or periodically at Kitware.

       \item Educators may wish to use OTB in courses. Materials are being
       developed for this purpose, e.g., a one-day, conference course and
       semester-long graduate courses. Watch the OTB web pages or check in
       the \code{OTB-Documents/CourseWare} directory for more information.
\end{itemize}

\section{A Brief History of OTB}
\label{sec:History}

\index{OTB!history}


Beside the Pleiades (PHR) and Cosmo-Skymed (CSK) systems developments forming ORFEO, the dual and bilateral system (France - Italy) for Earth Observation, the ORFEO Accompaniment Program was set up, to prepare, accompany and promote the use and the exploitation of the images derived from these sensors.

The creation of a preparatory program\footnote{http://smsc.cnes.fr/PLEIADES/A\_prog\_accomp.htm} is needed because of :
\begin{itemize}
\item the new capabilities and performances of the ORFEO systems (optical and radar high resolution, access capability, data quality, possibility to acquire simultaneously in optic and radar),
\item the implied need of new methodological developments : new processing methods, or adaptation of existing methods,
\item the need to realise those new developments in very close
  cooperation with the final users for better integration of new products in their systems.
  
\end{itemize}

This program was initiated by CNES mid-2003 and will last until 2009.
It consists in two parts, between which it is necessary to keep a strong interaction :
\begin{itemize}
\item A Thematic part
\item A Methodological part.
\end{itemize}

The Thematic part covers a large range of applications (civil and
defence ones), and aims at specifying and validating value added
products and services required by end users. This part includes
consideration about products integration in the operational systems or
processing lines. It also includes a careful thought on intermediary
structures to be developed to help non-autonomous users. Lastly, this part aims at raising future users awareness, through practical demonstrations and validations.

The Methodological part objective is the definition and the
development of tools for the operational exploitation of the future
submetric optic and radar images (tridimensional aspects, change
detection, texture analysis, pattern matching, optic radar
complementarities). It is mainly based on R\&D studies and doctorate
and post-doctorate research.

In this context, CNES\footnote{http://www.cnes.fr} decided to develop
the \emph{ORFEO ToolBox} (OTB), a set of algorithms encapsulated in a
software library. The goals of the OTB is to capitalise a methological
\textit{savoir faire} in order to adopt an incremental development
approach aimin to efficiently exploit the results obtained in the
frame of methodological R\&D studies.

All the developments are based on FLOSS (Free/Libre Open Source
Software) or existing CNES developments.

OTB is implemented in C++ and is mainly based on
ITK\footnote{http://www.itk.org} (Insight Toolkit):
\begin{itemize}
  \item ITK is used as the core element of OTB
  \item OTB classes inherit from ITK classes
  \item The software development procedure of OTB is strongly inspired
  from ITK's (Extreme Programming, test-based coding, Generic
  Programming, etc.)
  \item The documentation production procedure is the same as for ITK
  \item Several chapters of the Software Guide are litterally copied
  from ITK's Software Guide (with permission).
  \item Many examples are taken from ITK.
\end{itemize}

\subsection{ITK's history}
      
In 1999 the US National Library of Medicine of the National Institutes of
Health awarded six three-year contracts to develop an open-source
registration and segmentation toolkit, that eventually came to be known as
the Insight Toolkit (ITK) and formed the basis of the Insight Software
Consortium. ITK's NIH/NLM Project Manager was Dr. Terry Yoo, who coordinated the
six prime contractors composing the Insight consortium. These consortium
members included three commercial partners---GE Corporate R\&D, Kitware,
Inc., and MathSoft (the company name is now Insightful)---and three academic
partners---University of North Carolina (UNC), University of Tennessee (UT)
(Ross Whitaker subsequently moved to University of Utah), and University of
Pennsylvania (UPenn). The Principle Investigators for these partners were,
respectively, Bill Lorensen at GE CRD, Will Schroeder at Kitware, Vikram
Chalana at Insightful, Stephen Aylward with Luis Ib\'a\~nez at UNC (Luis is now
at Kitware), Ross Whitaker with Josh Cates at UT (both now at Utah), and
Dimitri Metaxas at UPenn (now at Rutgers). In addition, several
subcontractors rounded out the consortium including Peter Raitu at Brigham \&
Women's Hospital, Celina Imielinska and Pat Molholt at Columbia University,
Jim Gee at UPenn's Grasp Lab, and George Stetten at the University of
Pittsburgh.

In 2002 the first official public release of ITK was made
available. 

   	


