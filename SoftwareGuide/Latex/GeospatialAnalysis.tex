
\chapter{Geospatial analysis}\label{sec:GeospatialAnalysis}

PostGIS is a geospatial extension to PostgreSQL which allows very powerful 
geospatial analysis and storage possibilities. The ability for OTB to 
communicate to this kind of data base is of major interest.
Although other geospatial databases exist, PostGIS seems to be the 
more widely accepeted among the open source alternatives. A generic 
implementation of the OTB/PostGIS interface should make possible to 
switch data bases without major problem.

With this OTB-PostGIS interface, OTB will cover the whole information 
production chain from sensor model to spatial analysis.
All the functionnalities are not fully completed for now but the next 
paragraph shows a basic example wich allow to create a postGIS table
by using the generic implementation of the OTB/PostGIS interface.
  
\section{Reading from and Writing to Geospatial DBs}\label{sec:PostGISCreateTable}
\input{PostGISCreateTable.tex}

This basic example illustrates the ability to integrate postgreSQL/PostGIS transactions
inside OTB process.
There is lot more to come...

%\section{Performing GIS Operations}\label{sec:GISOperations}

%\section{Examples of Geospatial Analysis}

%\subsection{Building Extraction}
%\subsection{OSM Layer for the Viewer}
%\subsection{Urban area detection on OSM}
%\subsection{Road DB Update}
%\subsection{Counting Trees along Roads}



